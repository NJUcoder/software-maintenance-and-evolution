\documentclass[UTF8]{ctexart}

\title{软件重构}
\author{沈思远}

\begin{document}

\maketitle
\tableofcontents
\section{背景}
介绍重构这个概念产生的背景与动机
\section{重构的定义和案例介绍}
重构是一个在不改变代码外在行为的前提下,对代码作出修改,以改进程序的内部结构。重构是一种有纪律的、经过训练的、有条不紊的程序整理方法,可以将整理过程中不小心引入错误的几率降到最低。本质上说,重构就是在代码写好之后改进它的设计。然后设计一个JAVA的代码示例来介绍代码可以进行哪些重构
\section{代码的bad smell以及重构原则}
详细一些常见的具有bad smell的代码模式,并介绍如何通过重构技术来进行改进。最后总结一些重构的原则,比如:何时重构,如何处理重构对于系统设计以及运行性能的影响
\section{当前流行的自动化重构工具}
目前大多数重构工作都是需要程序员手动完成的,本章将会搜集一些当前流行的重构工具所使用的重构方法,并讨论如何进行使用
\section{重构方法的研究趋势}
介绍当前自动化重构工具的研究难点以及未解决的一些问题
\end{document}